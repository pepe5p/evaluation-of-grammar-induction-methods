\chapter{Algorytmy}
\label{cha:algorytmy}

Istnieje wiele algorytmów do odkrywania struktury języka formalnego. Na przestrzeni lat powstało również wiele podejść i metod, które pomagają tworzyć zarówno rozwiązania ogólne, jak i te szczególnie efektywne w swojej niszy. Z uwagi na dużą liczbę dostępnych algorytmów zdecydowałem się wybrać pewien ich podzbiór, który wykorzystam w pracy. Są to \textit{L*}, \textit{RPNI}, \textit{GIG} i \textit{ALERGIA}, opisane kolejno w sekcjach \ref{sec:l-star}, \ref{sec:rpni}, \ref{sec:gig} i \ref{sec:alergia}. To właśnie te algorytmy przedstawię jako pierwsze. Następnie krótko omówię pozostałe algorytmy, które uznałem za interesujące, oraz wyjaśnię, dlaczego nie zdecydowałem się na ich użycie.

\newenvironment{observationtable}[1][Tabela Obserwacji]{
  \begin{center}
  \renewcommand{\arraystretch}{1.3}
  \textbf{#1}\vspace{0.5em} \\
  \begin{tabular}{|c|c|c|}
  \hline
}{
  \hline
  \end{tabular}
  \end{center}
}

\section{Algorytm \textit{L*}}
\label{sec:l-star}

Algorytm \textit{L*} \cite{L_STAR} wyróżnia się na tle innych algorytmów do odkrywania gramatyki języka regularnego użytych w pracy dzięki swojemu podejściu do uczenia. Większość innych algorytmów opiera się wyłącznie na przykładach pozytywnych lub zarówno na przykładach pozytywnych, jak i negatywnych. Algorytm \textit{L*} wyłamuje się z tej konwencji, wykorzystując zapytania oraz kontrprzykłady do procesu nauki.

Model algorytmu składa się z dwóch aktorów: \textbf{nauczyciela} oraz \textbf{ucznia}. Jego celem jest skonstruowanie minimalnego deterministycznego automatu skończonego (DFA), który rozpoznaje dany język regularny na podstawie ograniczonych interakcji między dwoma uczestnikami procesu.

\subsection{Metoda}

Konstruowanie minimalnego automatu deterministycznego (DFA), który akceptuje język \( L \), odbywa się przy pomocy dwóch rodzajów zapytań:
\begin{itemize}
    \item \textbf{Zapytania o członkostwo (Membership Queries, \textit{MQ}):} Uczeń pyta, czy słowo \( w \) należy do języka \( L \). Nauczyciel odpowiada „tak” (\( w \in L \)) lub „nie” (\( w \notin L \)).
    \item \textbf{Zapytania o równoważność (Equivalence Queries, \textit{EQ}):} Uczeń przedstawia hipotezę \( H \), reprezentującą język \( L(H) \), i pyta, czy \( L(H) = L \). Jeśli hipoteza jest niepoprawna, nauczyciel dostarcza kontrprzykład \( w \), dla którego \( w \in L \) i \( w \notin L(H) \), lub \( w \notin L \) i \( w \in L(H) \).
\end{itemize}

Dla poprawnego działania algorytmu wymagane jest, aby nauczyciel mógł udzielać prawdziwych odpowiedzi na oba rodzaje zadawanych pytań. Nauczyciela często nazywa się również \textbf{wyrocznią}, co jest skrótem myślowym, ponieważ formalnie nauczyciel składa się z pary \textbf{wyroczni (oracles)}: \textit{MQ} i \textit{EQ}. Zazwyczaj to na użytkowniku algorytmu spoczywa odpowiedzialność za określenie nauczyciela, co wymaga już na wstępie posiadania pewnej wiedzy na temat danego języka. Jest to największe ograniczenie w zastosowaniu algorytmu \textit{L*}.

Proces algorytmu \textit{L*} opiera się na iteracyjnym konstruowaniu \textbf{tabeli obserwacji (Observation Table)}, która gromadzi informacje o języku L uzyskane na podstawie zapytań typu \textit{MQ}. Na podstawie zawartości tabeli formułowane są kolejne hipotezy dotyczące automatu DFA, które następnie poddawane są weryfikacji za pomocą zapytań typu \textit{EQ}.

Główne kroki algorytmu:
\begin{enumerate}
    \item \textbf{Inicjalizacja:} Algorytm rozpoczyna od utworzenia tabeli obserwacji wykorzystując minimalne zbiory prefiksów (\( S \)) i sufiksów (\( E \)).
    \item \textbf{Sprawdzanie domknięcia i spójności:} Algorytm weryfikuje, czy tabela obserwacji spełnia wymagania:
    \begin{itemize}
        \item \textbf{Domknięcie:} Każdy prefiks prowadzi do jednego z już znanych stanów.
        \item \textbf{Spójność:} Wiersze tabeli są rozróżnialne za pomocą sufiksów w \( E \).
    \end{itemize}
    Jeśli tabela nie spełnia tych wymagań algorytm zwiększa zbiory \( S \) i \( E \).
    \item \textbf{Tworzenie hipotezy:} Na podstawie tabeli obserwacji generowany jest automat DFA \( H \).
    \item \textbf{Test równoważności:} Hipoteza \( H \) jest testowana za pomocą \textit{EQ}. W przypadku błędu tabela jest aktualizowana na podstawie kontrprzykładu.
    \item \textbf{Terminacja:} Proces kończy się, gdy hipoteza \( H \) jest poprawna.
\end{enumerate}

\subsection{Formalizacja}

Celem tej sekcji jest przedstawienie formalnych podstaw i narzędzi matematycznych, które są niezbędne do zrozumienia algorytmu \( L^* \). W sekcji tej zawarto definicje kluczowych pojęć, takich jak języki regularne i automaty deterministyczne (DFA), które stanowią fundament problemu inferencji języka na podstawie ograniczonego zestawu danych.

\begin{definition}[DFA]  
    Skończony automat deterministyczny (DFA) to piątka uporządkowana \( M = (Q, \Sigma, \delta, q_0, F) \), gdzie:
    \begin{itemize}
        \item \( Q \): skończony zbiór stanów,
        \item \( \Sigma \): alfabet,
        \item \( \delta: Q \times \Sigma \to Q \): funkcja przejścia,
        \item \( q_0 \in Q \): stan początkowy,
        \item \( F \subseteq Q \): zbiór stanów akceptujących.
    \end{itemize}
\end{definition}

\begin{definition}[Akceptowanie słowa]
    Automat \( M \) akceptuje słowo \( w \in \Sigma^* \), jeśli \( \delta(q_0, w) \in F \), gdzie \( \delta(q_0, w) \) oznacza iteracyjne zastosowanie funkcji przejścia \( \delta \) do słowa \( w \).
\end{definition}

\begin{definition}[Język regularny]  
    Język \( L \subseteq \Sigma^* \) zdefiniowany nad alfabetem \( \Sigma \) nazywamy językiem regularnym, jeśli istnieje deterministyczny automat skończony (DFA) \( M = (Q, \Sigma, \delta, q_0, F) \), który akceptuje wszystkie słowa należące do \( L \), i tylko te słowa:
    \[
    L = \{ w \in \Sigma^* \mid M \text{ akceptuje } w \}.
    \]
\end{definition}

Formalizacja DFA pozwala przejść od intuicyjnego opisu języka regularnego do jego precyzyjnej, algebraicznej reprezentacji. Stanowi to punkt wyjścia do rozważania procesu konstrukcji automatu \( H \) na podstawie tabeli obserwacji.

\begin{definition}[Tabela obserwacji] 
    Tabela obserwacji (ObservationTable) to struktura danych definiowana jako trójka \( (S, E, T) \), gdzie:
    \begin{itemize}
        \item \( S \subseteq \Sigma^* \): skończony zbiór prefiksów,
        \item \( E \subseteq \Sigma^* \): skończony zbiór sufiksów,
        \item \( T: (S \cup S \cdot \Sigma) \times E \to \{0, 1\} \): funkcja wartościująca wynik zapytania \( MQ(s \cdot e) \), gdzie \( s \in S \cup S \cdot \Sigma \) i \( e \in E \).
    \end{itemize}
\end{definition}

Tabela obserwacji jest kluczowym elementem algorytmu \( L^* \). Przechowuje informacje o zachowaniu języka \( L \) względem słów w zbiorach \( S \) i \( E \). Dzięki temu pozwala grupować słowa w klasy równoważności, co jest podstawą do wyznaczenia stanów DFA.

\begin{definition}[Domknięcie] 
    Tabela obserwacji jest domknięta jeśli dla każdego \( s \in S \cdot \Sigma \), istnieje \( s' \in S \), taki że \( \forall e \in E, \, T(s, e) = T(s', e) \).
\end{definition}
\begin{definition}[Spójność]
    Tabela obserwacji jest spójna, jeśli dla dowolnych \( s_1, s_2 \in S \), spełniających warunek:
    \[
    \forall e \in E, \, T(s_1, e) = T(s_2, e),
    \]
    zachodzi również:
    \[
    \forall a \in \Sigma, \, \forall e \in E, \, T(s_1 \cdot a, e) = T(s_2 \cdot a, e).
    \]
\end{definition}

Warunki te zapewniają, że tabela obserwacji reprezentuje wystarczająco bogatą informację o języku \( L \), aby można było skonstruować poprawny DFA. \textbf{Domknięcie} gwarantuje, że wszystkie przejścia są uwzględnione w zbiorze \( S \), natomiast \textbf{spójność} zapewnia, że tabela jest zgodna z oczekiwanym zachowaniem języka.

\begin{definition}[Tworzenie hipotezy]
    Na podstawie tabeli (ObservationTable), która spełnia warunki \textbf{domknięcia} i \textbf{spójności}, można skonstruować DFA \( H = (Q, \Sigma, \delta, q_0, F) \), gdzie:
    \begin{itemize}
        \item \( Q = \{[s] \mid s \in S\} \), gdzie \( [s] \) oznacza klasę równoważności \( s \) w \( T \),
        \item \( \delta([s], a) = [s \cdot a] \),
        \item \( q_0 = [\epsilon] \),
        \item \( F = \{[s] \mid T(s, \epsilon) = 1\} \).
    \end{itemize}
\end{definition}

Klasy równoważności w \( T \) grupują słowa \( s \in S \) o identycznym zachowaniu względem języka \( L \). Dzięki temu DFA \( H \) przechodzi między stanami odpowiadającymi tym klasom, w oparciu o funkcję przejścia \( \delta \).

\begin{definition}[Kontrprzykład]
    Kontrprzykład dla hipotezy H odnośnie języka L nazywamy takie $w$, że \( w \in L \setminus L(H) \cup L(H) \setminus L \).
\end{definition}

\begin{definition}[Zapytanie o członkostwo]
    \emph{Zapytanie o członkostwo} (\emph{Membership Query}) to funkcja \( MQ: \Sigma^* \to \{0, 1\} \), która dla danego słowa \( w \in \Sigma^* \) zwraca \( 1 \), jeśli \( w \in L \), lub \( 0 \) w przeciwnym przypadku.
\end{definition}

\begin{definition}[Zapytanie o równoważność]
    \emph{Zapytanie o równoważność} (\emph{Equivalence Query}) to funkcja \( EQ: \mathcal{H} \to (\Sigma^* \cup \{\text{brak}\}) \), która dla hipotezy \( H \) zwraca:
    \begin{itemize}
        \item brak, jeśli \( L(H) = L \), czyli hipoteza \( H \) jest równoważna językowi \( L \).
        \item w przeciwnym wypadku \textbf{kontrprzykład},
    \end{itemize}
\end{definition}


Jeśli język generowany z hipotezy \( H \) nie jest równoważny \( L \), Nauczyciel dostarcza kontrprzykład \( w \). Prefiksy \( w \) są dodawane do \( S \), a tabela jest aktualizowana zgodnie z wynikami zapytań \( MQ \). Proces ten jest iteracyjny i pozwala na sukcesywne udoskonalanie tabeli obserwacji oraz hipotezy automatu \( H \), aż do uzyskania poprawnego modelu języka.

\subsection{Złożoność}

Algorytm \textit{L*} składa się z kilku etapów, których złożoność można szczegółowo przeanalizować. Złożoność czasowa algorytmu jest wielomianowa, co zostało udowodnione \cite{L_STAR}. Aby wyliczyć złożoność pamięciową trzeba zdefiniować:
\begin{itemize}
    \item \(n\) - liczba stanów szukanego minimalnego DFA,
    \item \(m\) - maksymalna długość kontrprzykładu,
    \item \(|\Sigma|\) - rozmiar alfabetu.
\end{itemize}

Struktura danych w algorytmie składa się z dwóch komponentów:
\begin{itemize}
    \item \textbf{Tabela obserwacji:}
    \begin{itemize}
        \item Zbiór \(S\) (prefiksy): maksymalnie \(O(n + m \cdot (n - 1))\) wpisów,
        \item Zbiór \(E\) (sufiksy): maksymalnie \(O(n)\) wpisów,
        \item Całkowity rozmiar tabeli: \(O(n \cdot m)\).
    \end{itemize}
    \item \textbf{DFA hipotezy:} DFA może mieć maksymalnie \(n\) stanów i \(O(n \cdot |\Sigma|)\) przejść.
\end{itemize}

Algorytm wymaga pamięci na tabelę obserwacji oraz strukturę DFA, stąd można obliczyć łączną złożoność pamięciową:
\[
O(n \cdot m + n \cdot |\Sigma|).
\]

Podsumowując czynnikami wpływającymi na złożoność pamięciową są:
\begin{itemize}
    \item \textbf{Rozmiar alfabetu (\(|\Sigma|\))}: Większy alfabet oznacza więcej przejść w tabeli obserwacji i DFA.
    \item \textbf{Długość kontrprzykładów (\(m\))}: Dłuższe kontrprzykłady zwiększają liczbę zapytań o członkostwo i rozmiar tabeli obserwacji.
    \item \textbf{Liczba stanów DFA (\(n\))}: Więcej stanów oznacza większą tabelę obserwacji i więcej iteracji algorytmu.
\end{itemize}

\subsection{Przykłady działania}

Aby rozjaśnić działanie algorytmu \textit{L*}, rozważmy proste języki $L_1$ i $L_2$

\[
L_1 = \{ w \in \{a, b\}^* \mid \text{liczba liter } a \text{ w } w \text{ jest parzysta} \}.
\]
\[
L_2 = \{ w \in \{a, b\}^* \mid \text{liczba liter } a \text{ jest parzysta lub liczba liter } b \text{ jest parzysta} \}.
\]

Minimalny DFA dla języka $L_1$ ma dwa stany:
\begin{itemize}
    \item Stan $s_0$: parzysta liczba $a$,
    \item Stan $s_1$: nieparzysta liczba $a$.
\end{itemize}

Dla języka $L_2$ zawiera cztery stany, ponieważ istnieją cztery możliwe kombinacje parzystości:
\begin{itemize}
    \item Stan $s_0$: parzysta liczba $a$ i parzysta liczba $b$,
    \item Stan $s_1$: nieparzysta liczba $a$ i parzysta liczba $b$,
    \item Stan $s_2$: nieparzysta liczba $a$ i nieparzysta liczba $b$,
    \item Stan $s_3$: parzysta liczba $a$ i nieparzysta liczba $b$.
\end{itemize}

Schematy minimalnych DFA można zobaczyć kolejno na rysunkach \ref{fig:dfa_even_a} i \ref{fig:dfa_even_a_or_even_b}.

\subsubsection{Przykład 1}

\paragraph*{Inicjalizacja tabeli obserwacji.}
Rozpoczynamy od tworzenia minimalnych zbiorów \textit{S} i \textit{E}:
\[
S = \{\epsilon\}, \quad E = \{\epsilon\}.
\]
Uczeń zadaje pytanie o członkostwo \( MQ(\epsilon) \), a Nauczyciel odpowiada „tak”, ponieważ \( \epsilon \in L \). Na tej podstawie konstruujemy tabelę \ref{tab:lang_1_observation_1}.

\begin{table}
    \centering
    \begin{tabular}{c|c}
        \diagbox{\( S \cup (S \cdot \Sigma) \)}{$E$} & \( \epsilon \) \\
        \hline
        $\epsilon$ & 1 \\
    \end{tabular}
    \caption{Tabela obserwacji $T_0$ dla języka \( L_1 \).}
    \label{tab:lang_1_observation_1}
\end{table}

\paragraph*{Dodanie nowych wierszy.}
Uczeń dodaje \( S \cdot \Sigma = \{a, b\} \) do tabeli i zadaje zapytania o członkostwo:
\[
MQ(a) = 0, \quad MQ(b) = 1.
\]
Zaktualizowaną tabelę widać w tabeli \ref{tab:lang_1_observation_2}.Tabela jest \textbf{niedomknięta}, ponieważ dla \(a\) nie istnieje takie \(s \in S \), że \( \forall e \in E, \, T(a, e) = T(s, e) \).

\begin{table}
    \centering
    \begin{tabular}{c|c}
        \diagbox{\( S \cup (S \cdot \Sigma) \)}{$E$} & \( \epsilon \) \\
        \hline
        $\epsilon$ & 1 \\
        \hline
        $a$              & 0 \\
        $b$              & 1 \\
    \end{tabular}
    \caption{Tabela obserwacji $T_1$ dla języka \( L_1 \).}
    \label{tab:lang_1_observation_2}
\end{table}

\paragraph*{Rozszerzenie \( S \).}
Dodajemy \( a \) do \( S \), aby tabela była domknięta:
\[
S = \{\epsilon, a\}.
\]
Uczeń zadaje zapytania dla nowych rozszerzeń \( S \cdot \Sigma \):
\[
MQ(aa) = 1, \quad MQ(ab) = 0.
\]
Nowe obserwacje widać w tabeli \ref{tab:lang_1_observation_3}. Tabela jest teraz \textbf{domknięta}. Jest też ona \textbf{spójna}. Możemy teraz skonstruować hipotezę.

\begin{table}
    \centering
    \begin{tabular}{c|c}
        \diagbox{\( S \cup (S \cdot \Sigma) \)}{$E$} & \( \epsilon \) \\
        \hline
        $\epsilon$      & 1 \\
        $a$             & 0 \\
        \hline
        $b$             & 1 \\
        $aa$            & 1 \\
        $ab$            & 0 \\
    \end{tabular}
    \caption{Tabela obserwacji $T_2$ dla języka \( L_1 \).}
    \label{tab:lang_1_observation_3}
\end{table}

\paragraph*{Konstrukcja hipotezy.}
Na podstawie tabeli obserwacji \ref{tab:lang_1_observation_3} konstruujemy hipotezę automatu DFA:
\begin{itemize}
    \item \textbf{Stany DFA:} \( q_\epsilon, q_a \), odpowiadające \( T(\epsilon, \epsilon) \) oraz \( T(a, \epsilon) \).
    \item \textbf{Przejścia:}
    \begin{align*}
        & q_\epsilon \xrightarrow{a} q_a, \quad q_\epsilon \xrightarrow{b} q_\epsilon, \\
        & q_a \xrightarrow{a} q_\epsilon, \quad q_a \xrightarrow{b} q_a.
    \end{align*}
    \item \textbf{Stan początkowy:} \( q_\epsilon \), ponieważ odpowiada wierszowi \( \epsilon \).
    \item \textbf{Stany akceptujące:} \( q_\epsilon \), ponieważ odpowiada \( T(\epsilon, \epsilon) = 1 \).
    \item \textbf{Stany nieakceptujące:} \( q_a \), ponieważ \( T(a, \epsilon) = 0 \).
\end{itemize}

\paragraph*{Testowanie hipotezy.}
Hipoteza jest testowana przez zapytanie o równoważność (EQ). Nauczyciel odpowiada "tak" co kończy działanie algorytmu. 

\paragraph*{Ostateczny automat.}
Automat DFA reprezentujący język o parzystej liczbie \( a \):
\[
\begin{array}{c|c|c}
\text{Stan} & a & b \\
\hline
\rightarrow * q_\epsilon & q_a & q_\epsilon \\
q_a & q_\epsilon & q_a \\
\end{array}
\]

\begin{figure}[ht]
    \centering
    \includegraphics[width=0.2\linewidth]{images/dfa_even_a.png}
    \caption{Minimalny DFA generujący język $L_1$.}
    \label{fig:dfa_even_a}
\end{figure}

\subsubsection{Przykład 2}

\paragraph*{Dodanie nowych wierszy.}
Inicjalizacja tabeli obserwacji przebiega tak samo jak w przykładzie pierwszym.
Uczeń rozszerza \( S \) o \( S \cdot \Sigma = \{a, b\} \) i zadaje pytania:
\[
MQ(a) = 0, \quad MQ(b) = 0.
\]
Tabelę po aktualizacji widać w tabeli \ref{tab:lang_2_observation_1}. Jest ona \textbf{spójna} i \textbf{domknięta}. Można, więc skonstruować hipotezę.

\begin{table}
    \centering
    \begin{tabular}{c|c}
        \diagbox{\( S \cup (S \cdot \Sigma) \)}{$E$} & \( \epsilon \) \\
        \hline
        $\epsilon$      & 1 \\
        \hline
        $a$             & 1 \\
        $b$             & 1 \\
    \end{tabular}
    \caption{Tabela obserwacji $T_1$ dla języka \( L_2 \).}
    \label{tab:lang_2_observation_1}
\end{table}

\paragraph*{Konstrukcja hipotezy $H_1$.}
Na podstawie tabeli obserwacji \ref{tab:lang_2_observation_1} konstruujemy hipotezę:
\begin{itemize}
    \item \textbf{Stany DFA:} \( q_\epsilon \), odpowiadający unikalnym wierszom tabeli.
    \item \textbf{Przejścia:}
    \begin{align*}
        & q_\epsilon \xrightarrow{a} q_\epsilon, \quad q_\epsilon \xrightarrow{b} q_\epsilon.
    \end{align*}
    \item \textbf{Stan początkowy:} \( q_\epsilon \).
    \item \textbf{Stany akceptujące:} \( q_\epsilon \).
\end{itemize}

\paragraph*{Testowanie hipotezy $H_1$.}
Hipoteza jest testowana przez zapytanie o równoważność (EQ). Można wybrać nieskończenie wiele kontrprzykładów, ale ja przyjmę, że nauczyciel zwraca $ba$.

\paragraph*{Aktualizacja tabeli obserwacji.}
Po otrzymaniu kontrprzykładu \( ba \), algorytm aktualizuje tabelę obserwacji. Wszystkie prefiksy \( ba \) są dodawane do zbioru \( S \), co prowadzi do:
\[
S = \{\epsilon, b, ba\}.
\]
Uczeń zadaje zapytania \( MQ \) dla nowych elementów \( S \cup (S \cdot \Sigma) \). Wyniki zapytań są następujące:
\[
MQ(ba) = 0, \quad MQ(baa) = 1, \quad MQ(bab) = 1.
\]
Na tej podstawie aktualizujemy tabelę obserwacji. Wynikową tabelę obserwacji pokazano w tabeli \ref{tab:lang_2_observation_2}. Jest ona \textbf{domknięta} i \textbf{niespójna}, ponieważ \( T(\epsilon, \epsilon) = T(b, \epsilon) \), a \( T(\epsilon \cdot a, \epsilon) \neq T(b \cdot a, \epsilon) \), stąd $a$ jest rozróżniającym sufiksem. Dodajemy $a$ do $E$ i uzupełniamy tabelę obserwacji \ref{tab:lang_2_observation_3}. Jest ona \textbf{domknięta} i \textbf{spójna}, więc możemy skonstruować hipotezę.
\[
MQ(\epsilon \cdot a) = 1, \quad MQ(b \cdot a) = 0, \quad MQ(ba \cdot a) = 1,
\]
\[
MQ(a \cdot a) = 1, \quad MQ(bb \cdot a) = 1, \quad MQ(baa \cdot a) = 0, \quad MQ(bab \cdot a) = 1.
\]

\begin{table}
    \centering
    \begin{tabular}{c|c}
        \diagbox{\( S \cup (S \cdot \Sigma) \)}{$E$} & \( \epsilon \) \\
        \hline
        $\epsilon$      & 1 \\
        $b$             & 1 \\
        $ba$            & 0 \\
        \hline
        $a$             & 1 \\
        $bb$            & 1 \\
        $baa$           & 1 \\
        $bab$           & 1 \\
    \end{tabular}
    \caption{Tabela obserwacji $T_2$ dla języka \( L_2 \).}
    \label{tab:lang_2_observation_2}
\end{table}

\begin{table}
    \centering
    \begin{tabular}{c|c|c}
        \diagbox{\( S \cup (S \cdot \Sigma) \)}{$E$} & \( \epsilon \) & $a$ \\
        \hline
        $\epsilon$      & 1 & 1 \\
        $b$             & 1 & 0 \\
        $ba$            & 0 & 1 \\
        \hline
        $a$             & 1 & 1 \\
        $bb$            & 1 & 1 \\
        $baa$           & 1 & 0 \\
        $bab$           & 1 & 1 \\
    \end{tabular}
    \caption{Tabela obserwacji $T_3$ dla języka \( L_2 \).}
    \label{tab:lang_2_observation_3}
\end{table}

\paragraph*{Konstrukcja hipotezy $H_2$.}
Na podstawie tabeli obserwacji \ref{tab:lang_2_observation_3} konstruujemy hipotezę:
\begin{itemize}
    \item \textbf{Stany DFA:} \( q_\epsilon, q_b, q_{ba} \), odpowiadający unikalnym wierszom tabeli.
    \item \textbf{Przejścia:}
    \begin{align*}
        & q_\epsilon \xrightarrow{a} q_\epsilon, \quad q_\epsilon \xrightarrow{b} q_b, \\
        & q_b \xrightarrow{a} q_{ba}, \quad q_b \xrightarrow{b} q_\epsilon, \\
        & q_{ba} \xrightarrow{a} q_\epsilon, \quad q_{ba} \xrightarrow{b} q_b.
    \end{align*}
    \item \textbf{Stan początkowy:} \( q_\epsilon \).
    \item \textbf{Stany akceptujące:} \( q_\epsilon, q_b \).
\end{itemize}

\paragraph*{Testowanie hipotezy $H_2$.}
Hipoteza jest testowana przez zapytanie o równoważność (EQ). Można wybrać nieskończenie wiele kontrprzykładów, ale ja przyjmę, że nauczyciel zwraca $ab$. Jest to kontrprzykład, ponieważ \( ab \in L(H) \land ab \notin L \). 

\paragraph*{Aktualizacja tabeli obserwacji.}
Po otrzymaniu kontrprzykładu \( ab \), algorytm aktualizuje tabelę obserwacji. Na początek dodajemy \( a \) oraz \( ab \) do zbioru \( S \), co prowadzi do:
\[
S = \{ \epsilon, a, b, ab, ba \}.
\]
Uczeń zadaje pytania \( MQ \) dla nowych elementów \( S \cup (S \cdot \Sigma) \). Wyniki zapytań są następujące:
\[
MQ(aa) = 1, \quad MQ(ab) = 0, \quad MQ(aba) = 1, \quad MQ(abb) = 1,
\]
\[
MQ(aa \cdot a) = 1, \quad MQ(ab \cdot a) = 1, \quad MQ(aba \cdot a) = 0, \quad MQ(abb \cdot a) = 1.
\]
Na tej podstawie aktualizujemy tabelę obserwacji. Wynikową tabelę obserwacji pokazano w tabeli \ref{tab:lang_2_observation_4}. Tabela \( T_4 \) jest \textbf{domknięta}, ale \textbf{niespójna}, ponieważ wiersz dla $\epsilon$ jest taki sam jak wiersz dla $a$, a wiersz dla \( \epsilon \cdot b \) jest inny niż wiersz dla \( a \cdot b \), stąd $b$ jest rozróżniającym sufiksem. Dodajemy $b$ do $E$ i uzupełniamy tabelę obserwacji \ref{tab:lang_2_observation_5}. 
Dodanie \( b \) do \( E \) pozwala naprawić niespójność, rozróżniając zachowanie stanów dla różnych sufiksów. Uczeń zadaje nowe zapytania \( MQ \) dla wszystkich \( S \cup (S \cdot \Sigma) \) i uzupełnia tabelę. Wyniki zapytań:
\[
MQ(\epsilon \cdot b) = 1, \quad MQ(a \cdot b) = 0, \quad MQ(b \cdot b) = 1, \quad MQ(ab \cdot b) = 1.
\]
Jest ona \textbf{domknięta} i \textbf{spójna}. 

\begin{table}
    \centering
    \begin{tabular}{c|c|c}
        \diagbox{\( S \cup (S \cdot \Sigma) \)}{$E$} & \( \epsilon \) & $a$ \\
        \hline
        $\epsilon$      & 1 & 1 \\
        $a$             & 1 & 1 \\
        $ab$            & 0 & 1 \\
        $b$             & 1 & 0 \\
        $ba$            & 0 & 1 \\
        \hline
        $aa$            & 1 & 1 \\
        $bb$            & 1 & 1 \\
        $aba$           & 1 & 0 \\
        $abb$           & 1 & 1 \\
        $baa$           & 1 & 0 \\
        $bab$           & 1 & 1 \\
    \end{tabular}
    \caption{Tabela obserwacji $T_4$ dla języka \( L_2 \) po dodaniu \( a \) i \( ab \) do \( S \).}
    \label{tab:lang_2_observation_4}
\end{table}

\begin{table}
    \centering
    \begin{tabular}{c|c|c|c}
        \diagbox{\( S \cup (S \cdot \Sigma) \)}{$E$} & $\epsilon$ & $a$ & $b$ \\
        \hline
        $\epsilon$      & 1 & 1 & 1 \\
        $a$             & 1 & 1 & 0 \\
        $ab$            & 0 & 1 & 1 \\
        $b$             & 1 & 0 & 1 \\
        $ba$            & 0 & 1 & 1 \\
        \hline
        $aa$            & 1 & 1 & 1 \\
        $bb$            & 1 & 1 & 1 \\
        $aba$           & 1 & 0 & 1 \\
        $abb$           & 1 & 1 & 0 \\
        $baa$           & 1 & 0 & 1 \\
        $bab$           & 1 & 1 & 0 \\
    \end{tabular}
    \caption{Tabela obserwacji $T_5$ dla języka \( L_2 \) po dodaniu \( b \) do \( E \).}
    \label{tab:lang_2_observation_5}
\end{table}

\paragraph*{Konstrukcja hipotezy $H_3$.}
Na podstawie tabeli obserwacji konstruujemy hipotezę DFA:
\begin{itemize}
    \item \textbf{Stany DFA:} \( q_\epsilon, q_a, q_b, q_{ab} \)
    \item \textbf{Przejścia:}
    \begin{align*}
        & q_\epsilon \xrightarrow{a} q_a, \quad q_\epsilon \xrightarrow{b} q_b, \\
        & q_a \xrightarrow{a} q_\epsilon, \quad q_a \xrightarrow{b} q_{ab}, \\
        & q_b \xrightarrow{a} q_{ab}, \quad q_b \xrightarrow{b} q_\epsilon, \\
        & q_{ab} \xrightarrow{a} q_b, \quad q_{ab} \xrightarrow{b} q_a.
    \end{align*}
    \item \textbf{Stan początkowy:} \( q_\epsilon \).
    \item \textbf{Stany akceptujące:} \( q_\epsilon, q_a, q_b \)
\end{itemize}

\paragraph*{Testowanie hipotezy $H_3$.}
Hipoteza jest testowana przez zapytanie o równoważność (EQ). Nauczyciel odpowiada "tak", co kończy działanie algorytmu.

\paragraph*{Ostateczny automat.}
Automat DFA reprezentujący język \( L \) ma następującą macierz przejść:
\[
\begin{array}{c|c|c}
\text{Stan} & a & b \\
\hline
\rightarrow * q_\epsilon & q_a & q_b \\
* q_a & q_\epsilon & q_{ab} \\
* q_b & q_{ab} & q_\epsilon \\
q_{ab} & q_b & q_a \\
\end{array}
\]

\begin{figure}[ht]
    \centering
    \includegraphics[width=0.2\linewidth]{images/dfa_even_a_or_even_b.png}
    \caption{Minimalny DFA dla języka \( L_2 \): parzystość \( a \) lub \( b \).}
    \label{fig:dfa_even_a_or_even_b}
\end{figure}

% Opcjonalnie zastosowania i rozszerzenia

\section{Algorytm \textit{RPNI}}
\label{sec:rpni}

Algorytm \textit{RPNI} (\textit{Regular Positive and Negative Inference}) \cite{RPNI} jest jednym z klasycznych algorytmów do odkrywania gramatyki języka regularnego. Wyróżnia się zdolnością do generalizacji na podstawie zbiorów przykładów pozytywnych oraz negatywnych, wykorzystując podejście łączenia stanów w drzewie akceptacji prefiksów (PTA).

Model algorytmu zakłada, że język docelowy jest regularny, a dostępne dane (przykłady pozytywne \( S^+ \) i negatywne \( S^- \)) są wystarczająco reprezentatywne, aby umożliwić skonstruowanie minimalnego deterministycznego automatu skończonego (DFA). Proces ten polega na iteracyjnym scalaniu stanów w PTA w celu zbudowania modelu akceptującego wszystkie przykłady pozytywne i odrzucającego negatywne.

\subsection{Metoda}

Algorytm RPNI pozwala na indukcję minimalnego deterministycznego automatu skończonego (DFA), który jest zgodny z danymi pozytywnymi (\( S^+ \)) i danymi negatywnymi (\( S^- \)). Jego działanie opiera się na kilku kluczowych krokach, które prowadzą do iteracyjnego generalizowania języka \( L \) w sposób kontrolowany.

\paragraph*{Opis algorytmu.}
Algorytm RPNI składa się z następujących etapów:
\begin{enumerate}
    \item \textbf{Budowa automatu drzewa prefiksów (PTA):}
        Zbiór \( S^+ \) jest używany do budowy automatu reprezentującego dokładnie wszystkie słowa z \( S^+ \). Automatyczny akceptor prefiksów dokładnie reprezentuje język \( L(S^+) \).
    \item \textbf{Iteracyjne łączenie stanów:}
        Algorytm przegląda pary stanów w PTA i sprawdza, czy można je połączyć bez naruszania zgodności z \( S^- \). Połączenie stanów prowadzi do generalizacji języka \( L \).
    \item \textbf{Weryfikacja zgodności z \( S^- \):}
        Dla każdej potencjalnej zmiany algorytm sprawdza, czy żadne słowo z \( S^- \) nie jest akceptowane przez zmodyfikowany automat.
    \item \textbf{Zakończenie:} 
        Algorytm kończy działanie, gdy żadne dodatkowe pary stanów nie mogą być połączone. Wynikowy automat jest minimalny i deterministyczny.
\end{enumerate}

\paragraph*{Zalety algorytmu RPNI:}
\begin{itemize}
    \item \textbf{Minimalizacja DFA:} Wynikowy automat jest zawsze minimalny, zgodny z \( S^+ \) i \( S^- \).
    \item \textbf{Efektywność:} Algorytm działa w czasie wielomianowym względem rozmiaru danych wejściowych.
    \item \textbf{Zbieżność identyfikacji (identification in the limit):} RPNI gwarantuje poprawne działanie dla klas języków regularnych w przypadku wystarczająco dużego zestawu danych.
\end{itemize}

\subsection{Formalizacja}

Celem niniejszej sekcji jest dostarczenie formalnych definicji oraz narzędzi matematycznych niezbędnych do pełnego zrozumienia algorytmu RPNI. W szczególności omówione zostaną kluczowe koncepcje, takie jak deterministyczny automat skończony (DFA) oraz akceptor drzewa prefiksowego (Prefix Tree Acceptor, PTA), które stanowią podstawę teoretyczną algorytmu. Formalizacja ta pozwala na przejście od zestawu przykładów pozytywnych i negatywnych do precyzyjnego opisu języków regularnych oraz konstrukcji ich reprezentacji w postaci minimalnych automatów. Ponownie potrzebne nam są: definicja \ref{def:dfa} (DFA), definicja \ref{def:word_acceptance} (akceptowanie słowa) i definicja \ref{def:regular_language} (język regularny).

\begin{definition}[Zbiór pozytywny $S^+$]
    Zbiorem pozytywnym lub zbiorem przykładów pozytywnych nazywamy zbiór takich słów $w$, że \( w \in L \).
\end{definition}

\begin{definition}[Zbiór negatywny $S^-$]
    Zbiorem negatywnym lub zbiorem przykładów negatywnych nazywamy zbiór takich słów $w$, że \( w \notin L \).
\end{definition}

Oczywistą, ale wartą odnotowania implikacją jest to, że zbiór przykładów pozytywnych i zbiór przykładów negatywnych są rozłączne (\( S^+ \cap S^- = \emptyset \)).

\begin{definition}[Akceptor drzewa prefiksów (Prefix Tree Acceptor, PTA)]  
    Dla zbioru \( S^+ \), akceptor drzewa prefiksów \( PTA(S^+) \) to deterministyczny automat skończony \( M = (Q, \Sigma, \delta, q_0, F) \), gdzie:
    \begin{itemize}
        \item \( Q \): zbiór wszystkich prefiksów słów z \( S^+ \),
        \item \( \Sigma \): alfabet słów w \( S^+ \),
        \item \( \delta(q, a) = qa \): przejście w drzewie jest zdefiniowane przez konkatenację z symbolem \( a \),
        \item \( q_0 = \epsilon \): stan początkowy reprezentuje pusty prefiks,
        \item \( F = \{q \in Q \mid q \text{ jest pełnym słowem w } S^+\}\).
    \end{itemize}
\end{definition}

\begin{definition}[Łączenie stanów]
    \label{def:state_merging}
    Łączenie dwóch stanów \( q_1, q_2 \in Q \) w deterministycznym automacie skończonym \( M = (Q, \Sigma, \delta, q_0, F) \) polega na utworzeniu nowego automatu \( M' = (Q', \Sigma, \delta', q_0', F') \), gdzie zbiór stanów po łączeniu zawiera \( q_1 \) jako reprezentanta obu stanów, a \( q_2 \) jest usunięty. Ponadto funkcja przejścia jest odpowiednio zaktualizowana, aby przekierować wszystkie przejścia z \( q_2 \) na \( q_1 \). Jeśli \( q_2 \) był stanem akceptującym, \( q_1 \) również staje się stanem akceptującym.
    \begin{itemize}
        \item \( Q' = Q \setminus \{q_2\} \),
        \item \( \delta'(q, a) = 
        \begin{cases} 
            q_1 & \text{jeśli} \space \delta(q, a) = q_2, \\
            \delta(q, a) & \text{w przeciwnym przypadku,}
        \end{cases}
        \),
        \item \( F' =
        \begin{cases} 
            F & \text{jeśli } q_2 \notin F \\
            F \setminus \{q_2\} & \text{jeśli } q_1 \in F \land q_2 \in F \\
            (F \setminus \{q_2\}) \cup \{q_1\} & \text{jeśli } q_1 \notin F \land q_2 \in F \\
        \end{cases}
        \).
    \end{itemize}
\end{definition}

\paragraph*{Formalny proces algorytmu RPNI.}
\begin{itemize}
    \item \textbf{Konstrukcja \( PTA(S^+) \):} Utwórz akceptor drzewa prefiksów na podstawie \( S^+ \).
    \item \textbf{Iteracyjne łączenie stanów:} 
    \begin{itemize}
        \item Sprawdzaj każdą parę stanów \( q_1, q_2 \in Q \) w \( PTA(S^+) \).
        \item Połącz stany zgodnie z definicją \ref{def:state_merging}.
    \end{itemize}
    \item \textbf{Zakończenie:} Proces kończy się, gdy żadne kolejne łączenie stanów nie jest możliwe.
\end{itemize}

\paragraph*{Wynik:}
Rezultatem algorytmu jest minimalny DFA \( M \), który akceptuje wszystkie słowa z \( S^+ \) i żadne słowo z \( S^- \). Dzięki iteracyjnemu łączeniu stanów \( PTA(S^+) \) zostaje stopniowo uproszczony do ostatecznej postaci minimalnego automatu.

\subsection{Złożoność}

Algorytm \textit{RPNI} charakteryzuje się złożonością wielomianową względem rozmiaru danych wejściowych. Aby dokładnie określić jego złożoność czasową i pamięciową, należy zdefiniować kluczowe parametry:
\begin{itemize}
    \item \(|S^+|\) - liczba słów w zbiorze pozytywnych przykładów,
    \item \( l^+ \) - maksymalna długość słowa w \( S^+ \),
    \item \(|S^-|\) - liczba słów w zbiorze negatywnych przykładów,
    \item \( l^- \) - maksymalna długość słowa w \( S^- \),
    \item \(|\Sigma|\) - rozmiar alfabetu.
\end{itemize}

\paragraph*{Złożoność czasowa}
Algorytm RPNI składa się z dwóch głównych etapów:
\begin{enumerate}
    \item \textbf{Konstrukcja drzewa prefiksowego (PTA):}
    Utworzenie PTA wymaga przejścia przez wszystkie słowa w \( S^+ \), co ma złożoność \( O(|S^+| \cdot l^+) \).
    \item \textbf{Iteracyjne łączenie stanów:}
    Liczba stanów w \( PTA(S^+) \) jest równa liczbie wszystkich prefiksów słów w \( S^+ \), co w najgorszym przypadku wynosi \( O(|S^+| \cdot l^+) \). Każde łączenie dwóch stanów wymaga sprawdzenia, czy wynikowy automat jest zgodny z danymi negatywnymi \( S^- \). Sprawdzenie tego wymaga przejścia przez \( |S^-| \) słów. Etap ten ma, więc złożoność \( O((|S^+| \cdot l)^2 \cdot |S^-|) \).
\end{enumerate}

Łączna złożoność czasowa algorytmu wynosi:
\[
O(|S^+| \cdot l^+ + (|S^+| \cdot l^+)^2 \cdot |S^-|),
\]
co można uprościć do:
\[
O((|S^+| \cdot l^+)^2 \cdot |S^-|).
\]


\paragraph*{Złożoność pamięciowa}
Algorytm przechowuje następujące struktury danych:
\begin{itemize}
    \item \textbf{Drzewo prefiksowe (PTA):}  
    PTA zawiera \( O(|S^+| \cdot l^+) \) stanów, z których każdy może mieć do \( |\Sigma| \) przejść. Wymagana pamięć wynosi:
    \[
    O(|S^+| \cdot l^+ \cdot |\Sigma|).
    \]

    \item \textbf{Negatywne przykłady (\( S^- \)):}  
    Przechowywane są do weryfikacji zgodności połączeń. Przechowywanie \( |S^-| \) słów o maksymalnej długości \( l^- \) wymaga:
    \[
    O(|S^-| \cdot l^-).
    \]
\end{itemize}

Łączna złożoność pamięciowa algorytmu wynosi:
\[
O(|S^+| \cdot l^+ \cdot |\Sigma| + |S^-| \cdot l^-).
\]

\paragraph*{Czynniki wpływające na złożoność}
Główne elementy mające wpływ na wydajność algorytmu RPNI to:
\begin{itemize}
    \item \textbf{Rozmiar zbioru pozytywnych przykładów (\( S^+ \)):} Większy zbiór przykładów pozytywnych prowadzi do większego drzewa prefiksowego oraz większej liczby stanów do połączenia.
    \item \textbf{Rozmiar zbioru negatywnych przykładów (\( S^- \)):} Większy zbiór przykładów negatywnych zwiększa liczbę weryfikacji podczas łączenia stanów.
    \item \textbf{Długość słów ($l^+$ i $l^-$):} Dłuższe słowa zwiększają zarówno liczbę stanów w \( PTA \), jak i czas przetwarzania weryfikacji.
    \item \textbf{Rozmiar alfabetu (\( |\Sigma| \)):} Większy alfabet zwiększa liczbę przejść w \( PTA \).
\end{itemize}

\subsection{Przykład działania}

Poniżej przedstawiono krok po kroku przykład działania algorytmu RPNI. Rozważmy język $L$, który składa się ze słów zawierających literę $a$: 
\[
L = \{ w \in \{a, b, c\}^*a\{a, b, c\}^* \},
\]
oraz przyjmijmy następujące zbiory zdań pozytywnych i negatywnych:
\[
S^+ = \{ a, aa, ab, ba, aba, aab, baa, abaaa, baaaa, ac \},
\]
\[
S^- = \{ \epsilon, b, bb, c, bc, bbb, bbbc \}.
\]

Na początku algorytm konstruuje drzewo prefiksowe (\textit{Prefix Tree Acceptor}, PTA) na podstawie zbioru \( S^+ \). Struktura PTA jest zaprezentowana na \textbf{Rys. \ref{fig:pta00}}. Następnie algorytm iteracyjnie łączy stany w taki sposób, aby zachować zgodność zarówno z danymi negatywnymi (\( S^- \)). 

Warto podkreślić, że kolejność wykonywanych operacji łączenia zależy od konkretnej implementacji algorytmu. Istotne jednak jest to, że niezależnie od przyjętej kolejności wynik końcowy pozostaje taki sam. Dla celów ilustracyjnych stany do połączenia będą wybierane przez autora.

W pierwszej iteracji rozważamy połączenie stanów \( q_0 \) i \( q_1 \). Próba ich scalania nie może zakończyć się powodzeniem, ponieważ naruszyłaby zgodność z danymi negatywnymi (\( S^- \)). Przykładem takiego naruszenia jest słowo \( \epsilon \), które po połączeniu \( q_0 \) i \( q_1 \) zostałoby niesłusznie zaakceptowane. 

W kolejnej iteracji analizujemy stany \( q_1 \) i \( q_2 \). Ich połączenie jest możliwe, ponieważ nie narusza zgodności z danymi negatywnymi (\( S^- \)). W wyniku tej operacji scalania powstaje zaktualizowany graf, przedstawiony na \textbf{Rys. \ref{fig:pta01}}.

\begin{figure}[ht]
    \centering
    \includegraphics[width=0.8\textwidth]{images/pta00.png}
    \caption{Drzewo prefiksowe (PTA) skonstruowane dla \( S^+ \).}
    \label{fig:pta00}
\end{figure}

\begin{figure}[ht]
    \centering
    \includegraphics[width=0.8\textwidth]{images/pta01.png}
    \caption{Połączenie stanów \( q1 \) i \( q2 \).}
    \label{fig:pta01}
\end{figure} 

Następne pary stanów będą tak wyznaczane, aby były to poprawne łączenia w celu usprawnienia pokazywania przykładu. Kolejnym krokiem jest połączenie stanów \( q_3 \) i \( q_7 \), co przedstawiono na \textbf{Rys. \ref{fig:pta02}}. Algorytm następnie łączy stany \( q_3 \) i \( q_{13} \), co przedstawiono na \textbf{Rys. \ref{fig:pta03}}. Połączenie stanów \( q_3 \) i \( q_5 \) prowadzi do zaktualizowanego grafu widocznego na \textbf{Rys. \ref{fig:pta04}}. 

W tym momencie łatwo można zauważyć podobieństwo dwóch gałęzi wychodzących z węzła \( q_3 \). Stąd, w tym kroku algorytm łączy pary stanów: \( q_6 \) i \( q_8 \), \( q_9 \) i \( q_{11} \), oraz \( q_{10} \) i \( q_{12} \). Zaktualizowany graf przedstawiono na \textbf{Rys. \ref{fig:pta05}}. 

Następnie zostaje przeprowadzone łączenie stanów \( q_9 \) i \( q_{10} \), \( q_6 \) i \( q_9 \), oraz \( q_3 \) i \( q_6 \). Warte odnotowania jest to, że po raz pierwszy łączymy stan akceptujący z nieakceptującym. W takim przypadku nowy stan będzie akceptujący. Jest to generalizacja, która jest poprawna, ponieważ zawsze sprawdzamy zgodność ze zbiorem \( S^- \).

\begin{figure}[ht]
    \centering
    \includegraphics[width=0.8\textwidth]{images/pta02.png}
    \caption{Połączenie stanów \( q3 \) i \( q7 \).}
    \label{fig:pta02}
\end{figure}

\begin{figure}[ht]
    \centering
    \includegraphics[width=0.8\textwidth]{images/pta03.png}
    \caption{Połączenie stanów \( q3 \) i \( q13 \).}
    \label{fig:pta03}
\end{figure}

\begin{figure}[ht]
    \centering
    \includegraphics[width=0.8\textwidth]{images/pta04.png}
    \caption{Połączenie stanów \( q3 \) i \( q5 \).}
    \label{fig:pta04}
\end{figure}

\begin{figure}[ht]
    \centering
    \includegraphics[width=0.8\textwidth]{images/pta05.png}
    \caption{Połączenie stanów \( q6 \) i \( q8 \), \( q9 \) i \( q11 \), oraz \( q10 \) i \( q12 \).}
    \label{fig:pta05}
\end{figure}

Algorytm wykonuje kolejne połączenia stanów, co prowadzi do grafu widocznego na \textbf{Rys. \ref{fig:pta06}}. W tym kroku stany \( q_0 \) i \( q_4 \) zostają połączone. Wynik przedstawiono na \textbf{Rys. \ref{fig:pta07}}. W ostatnim kroku algorytm łączy stany \( q_1 \) i \( q_3 \), uzyskując finalny graf przedstawiony na \textbf{Rys. \ref{fig:pta08}}. 

Nie jest to jednak koniec, ponieważ brakuje przejścia \( c \) ze stanu \( q_0 \). W takim wypadku konieczne jest dodanie nowego cyklicznego przejścia, czyli z \( q_0 \) do \( q_0 \). Po tej operacji uzyskujemy wynikowy DFA, który generuje język \( L \).

\begin{figure}[ht]
    \centering
    \includegraphics[width=0.4\textwidth]{images/pta06.png}
    \caption{Połączenie stanów \( q9 \) i \( q10 \), \( q6 \) i \( q9 \), oraz \( q3 \) i \( q6 \).}
    \label{fig:pta06}
\end{figure}

\begin{figure}[ht]
    \centering
    \includegraphics[width=0.4\textwidth]{images/pta07.png}
    \caption{Połączenie stanów \( q0 \) i \( q4 \).}
    \label{fig:pta07}
\end{figure}

\begin{figure}[ht]
    \centering
    \includegraphics[width=0.4\textwidth]{images/pta08.png}
    \caption{Finalny graf po połączeniu stanów \( q1 \) i \( q3 \).}
    \label{fig:pta08}
\end{figure}

\section{Algorytm \textit{GIG}}
\label{sec:gig}

\section{Algorytm \textit{ALERGIA}}  
\label{sec:alergia}  

Algorytm \textit{ALERGIA} \cite{ALERGIA} jest stosowany do indukcji stochastycznych gramatyk regularnych. W przeciwieństwie do \textit{RPNI} lub \textit{GIG}, które zakładają obecność zarówno przykładów pozytywnych, jak i negatywnych, \textit{ALERGIA} opiera się wyłącznie na zbiorze przykładów pozytywnych. Umożliwia to jego zastosowanie w sytuacjach, gdzie dostęp do przykładów negatywnych jest ograniczony lub niemożliwy.  

Algorytm ten konstruuje deterministyczny stochastyczny automat skończony (DSFA) na podstawie drzewa akceptacji prefiksów (PTA) utworzonego z dostępnych danych. Proces uogólniania polega na łączeniu stanów automatu, przy jednoczesnym zachowaniu zgodności ze statystycznym rozkładem prawdopodobieństwa przejść między stanami. Decyzje o scalaniu są podejmowane na podstawie testów statystycznych, które zapewniają, że wynikowy model zachowuje probabilistyczne własności danych wejściowych.  

Jedną z kluczowych cech algorytmu \textit{ALERGIA} jest jego zdolność do indukcji gramatyk stochastycznych, co czyni go szczególnie przydatnym w zastosowaniach wymagających modelowania niepewności lub probabilistycznych wzorców zachowań. Dzięki swojej efektywności obliczeniowej algorytm ten znajduje zastosowanie w zadaniach takich jak rozpoznawanie wzorców, analiza sekwencji biologicznych oraz modelowanie języka naturalnego.

Algorytm działa iteracyjnie, testując kompatybilność stanów i łącząc te, które są statystycznie zgodne. Proces kończy się, gdy żadne dalsze scalenie nie jest możliwe bez naruszenia kryteriów statystycznych. Dzięki temu wynikowy automat nie tylko reprezentuje język generowany przez dane treningowe, ale także umożliwia estymację prawdopodobieństwa generacji nowych ciągów.


\subsection{Metoda}  
Algorytm \textit{ALERGIA} pozwala na indukcję deterministycznego stochastycznego automatu skończonego (DSFA), który przybliża rozkład prawdopodobieństwa nad danymi wejściowymi. Proces ten obejmuje następujące kroki:  

\paragraph*{Opis algorytmu.}  
Algorytm \textit{ALERGIA} składa się z kilku etapów:  
\begin{enumerate}  
    \item \textbf{Budowa drzewa prefiksów (PTA):}  
        Zbiór przykładów pozytywnych jest używany do zbudowania deterministycznego automatu reprezentującego dokładnie wszystkie sekwencje w danych.  
    \item \textbf{Iteracyjne łączenie stanów:}  
        Algorytm analizuje pary stanów w PTA i testuje ich zgodność statystyczną. Łączenie stanów jest realizowane tylko wtedy, gdy nie narusza statystycznej zgodności z danymi.  
    \item \textbf{Sprawdzanie zgodności:}  
        Kompatybilność stanów jest oceniana za pomocą testu Hoeffding'a oraz dodatkowego testu dla rozkładów wykładniczych czasów przejść.  
    \item \textbf{Zakończenie:}  
        Algorytm kończy działanie, gdy żadne dalsze scalenie stanów nie jest możliwe bez naruszenia zgodności statystycznej. Ostateczny automat aproksymuje rozkład probabilistyczny danych wejściowych.  
\end{enumerate}


\subsection{Formalizacja}  
W tej sekcji przedstawiono matematyczne podstawy algorytmu \textit{ALERGIA} oraz definicje kluczowych pojęć, które umożliwiają precyzyjne opisanie jego działania. Algorytm ten wykorzystuje akceptor drzewa prefiksów (PTA) jako strukturę początkową z definicji \ref{def:pta}. Proces formalizacji skupia się na mechanizmach scalania stanów, które pozwalają na stopniowe upraszczanie automatu przy jednoczesnym zachowaniu zgodności z probabilistycznymi wzorcami danych wejściowych.  

\begin{definition}[DSFA]  
\label{def:dsfa}
Stochastyczny, skończony automat deterministyczny to piątka uporządkowana
\[ 
A = (Q, \Sigma, \delta, q_0, P), 
\]
gdzie:  
\begin{itemize}  
    \item \( Q \) – skończony zbiór stanów,  
    \item \( \Sigma \) – alfabet,  
    \item \( \delta : Q \times \Sigma \to Q \) – funkcja przejścia,  
    \item \( q_0 \in Q \) – stan początkowy,  
    \item \( P: Q \times \Sigma \to [0, 1] \) – funkcja prawdopodobieństwa przejść, spełniająca:  
    \[
    \sum_{a \in \Sigma} P(q, a) \leq 1, \quad \forall q \in Q, \quad a \in \Sigma.
    \]  
\end{itemize}
\end{definition}

Automat \( A \) generuje język stochastyczny, w którym każda sekwencja symboli z \( \Sigma \) jest akceptowana z określonym prawdopodobieństwem, wyznaczanym przez funkcję \( P \).

\begin{definition}[Zgodność stanów]  
\label{def:alergia_state_compatibility}
Dwa stany \( q_1, q_2 \in Q \) są uznawane za zgodne (kompatybilne) i mogą zostać scalone, jeżeli spełnione są dwa warunki:  
\begin{enumerate}
    \item Funkcja prawdopodobieństwa przejść \( P \) dla każdego symbolu \( a \in \Sigma \) spełniają:  
    \[
    |P(q_1, a) - P(q_2, a)| \leq \epsilon,
    \]  
    gdzie \( \epsilon \) jest granicą wyznaczoną przez test Hoeffding’a:  
    \[
    \epsilon = \sqrt{\frac{1}{2} \ln\left(\frac{2}{\alpha}\right)} \left( \frac{1}{\sqrt{n_1}} + \frac{1}{\sqrt{n_2}} \right),
    \]  
    przy czym:  
    \begin{itemize}  
        \item \( n_1 \) – liczba obserwacji dla stanu \( q_1 \),  
        \item \( n_2 \) – liczba obserwacji dla stanu \( q_2 \),  
        \item \( \alpha \) – poziom ufności.  
    \end{itemize}
    \item Prawdopodobieństwa terminacji \( F \) spełniają:
    \[
    |F(q_1) - F(q_2)| \leq \epsilon,
    \]  
\end{enumerate}
\end{definition}

Test Hoeffding’a to statystyczna metoda oceny zgodności dwóch rozkładów prawdopodobieństwa, bazująca na nierówności Hoeffding’a. W kontekście algorytmu \textit{ALERGIA} test ten służy do sprawdzania, czy różnice między obserwowanymi częstościami przejść w stanach automatu można uznać za przypadkowe, czy też są one statystycznie istotne.  

Nierówność Hoeffding’a zapewnia górne ograniczenie na prawdopodobieństwo, że rzeczywiste wartości oczekiwane dwóch zmiennych losowych odbiegają od ich wartości estymowanych o więcej niż zadana tolerancja \( \epsilon \). 

W algorytmie \textit{ALERGIA} test Hoeffding’a porównuje rozkłady przejść pomiędzy stanami automatu. Jeśli różnice są mniejsze niż wyznaczony próg tolerancji, stany są uznawane za zgodne statystycznie i mogą zostać scalone.  

\begin{definition}[Łączenie stanów w \textit{ALERGIA}]  
\label{def:alergia_state_merging}  
Łączenie dwóch stanów \( q_1, q_2 \in Q \) w stochastycznym, skończonym automacie deterministycznym (DSFA) \( A = (Q, \Sigma, \delta, q_0, P) \) polega na utworzeniu nowego automatu \( A' = (Q', \Sigma, \delta', q_0, P') \), gdzie stan \( q_1 \) reprezentuje połączone stany, a \( q_2 \) zostaje usunięty. Funkcje przejścia i prawdopodobieństwa są aktualizowane zgodnie z poniższymi regułami:  
\begin{itemize}  
    \item \( Q' = Q \setminus \{q_2\} \),  
    \item \( \delta'(q, a) =  
    \begin{cases}  
        \delta(q_1, a) & \text{jeśli } \delta(q, a) = q_2, \\  
        \delta(q, a) & \text{w przeciwnym przypadku.}  
    \end{cases}  
    \)  
    \item Prawdopodobieństwa przejść są uśredniane w następujący sposób:  
    \[
    P'(q, a) = \frac{n_1 \cdot P(q_1, a) + n_2 \cdot P(q_2, a)}{n_1 + n_2},
    \]  
    gdzie \( n_1 \) i \( n_2 \) to liczby obserwacji w stanach \( q_1 \) i \( q_2 \).  
\end{itemize}  
\end{definition}  

Łączenie stanów jest przeprowadzane tylko wtedy, gdy dwa stany są zgodne według definicji \ref{def:alergia_state_compatibility}. Po scaleniu stanów \( q_1 \) i \( q_2 \) nowy stan dziedziczy przejścia oraz rozkłady prawdopodobieństwa jako uśrednienie. Scalanie w algorytmie kończy się, gdy żadne dalsze połączenia nie spełniają powyższych kryteriów zgodności.


\subsection{Złożoność}  
Algorytm \textit{ALERGIA} cechuje się złożonością wielomianową, zależną od rozmiaru danych wejściowych. W analizie złożoności czasowej i pamięciowej uwzględniamy następujące parametry:  
\begin{itemize}  
    \item \(|S^+|\) - liczba przykładów pozytywnych,  
    \item \( l^+ \) - maksymalna długość ciągu w \( S^+ \),  
    \item \(|\Sigma|\) - liczba symboli w alfabecie.  
\end{itemize}  

\paragraph*{Złożoność czasowa}  
Algorytm \textit{ALERGIA} składa się z dwóch głównych etapów:  
\begin{enumerate}  
    \item \textbf{Budowa drzewa prefiksowego (PTA):}  
    Tworzenie drzewa wymaga przetworzenia wszystkich słów w zbiorze \( S^+ \), analizując je symbol po symbolu. Koszt tego kroku wynosi:  
    \[
    O(|S^+| \cdot l^+).
    \]  

    \item \textbf{Scalanie stanów:}  
    Każda para stanów w PTA jest analizowana pod kątem zgodności. Liczba stanów wynosi \( O(|S^+| \cdot l^+) \), co prowadzi do \( O((|S^+| \cdot l^+)^2) \) możliwych par do porównania.  

    Ponadto, dla każdej pary stanów sprawdzana jest zgodność ich rozkładów prawdopodobieństwa, co wymaga przeprowadzenia testów statystycznych (np. testu Hoeffding’a) dla wszystkich symboli w alfabecie \( \Sigma \). Złożoność tego kroku wynosi:  
    \[
    O(|\Sigma|).
    \]
\end{enumerate}  

\paragraph*{Łączna złożoność czasowa}  
Po uwzględnieniu obu etapów oraz kosztów związanych z rekurencyjnym sprawdzaniem zgodności, całkowita złożoność czasowa algorytmu \textit{ALERGIA} w najgorszym przypadku wynosi:  
\[
O((|S^+| \cdot l^+)^2 \cdot |\Sigma|).
\]

\paragraph*{Złożoność pamięciowa}  
Algorytm przechowuje następujące struktury:  
\begin{itemize}  
    \item \textbf{Drzewo prefiksowe (PTA):}  
    PTA składa się z \( O(|S^+| \cdot l^+) \) stanów, z których każdy może mieć do \( |\Sigma| \) przejść. Wymagana przestrzeń pamięci wynosi:  
    \[
    O(|S^+| \cdot l^+ \cdot |\Sigma|).
    \]  

    \item \textbf{Statystyki przejść:}  
    Dla każdego stanu przechowywane są rozkłady prawdopodobieństw oraz dane statystyczne, co daje dodatkowy koszt pamięciowy:  
    \[
    O(|S^+| \cdot l^+).
    \]  
\end{itemize}  

Łączna złożoność pamięciowa wynosi:  
\[
O(|S^+| \cdot l^+ \cdot |\Sigma|).
\]  

\paragraph*{Czynniki wpływające na złożoność}  
Efektywność algorytmu \textit{ALERGIA} zależy przede wszystkim od liczby przykładów pozytywnych (\( S^+ \)) oraz długości słów (\( l^+ \)), które determinują rozmiar drzewa prefiksowego i liczbę stanów do analizy. Dodatkowo, większy alfabet (\( |\Sigma| \)) zwiększa liczbę możliwych przejść, co wpływa zarówno na czas obliczeń, jak i wymagania pamięciowe. 

\subsection{Przykład działania}  
Aby zilustrować działanie algorytmu \textit{ALERGIA}, rozważmy prosty przykład danych wejściowych opisujących sekwencje wyników rzutów monetą. Każdy wynik jest oznaczony symbolem \( H \) (orzeł) lub \( T \) (reszka).

\paragraph*{Dane wejściowe.}  
Zbiór przykładów pozytywnych (\( S^+ \)):  
\[
S^+ = \{H, H, T, HH, HT, TH, TT, HHH\}.
\]  

\paragraph*{Krok 1: Konstrukcja drzewa prefiksowego (PTA).}  
Na podstawie danych pozytywnych algorytm buduje drzewo prefiksowe (Prefix Tree Acceptor, PTA), które reprezentuje dokładnie wszystkie obserwowane sekwencje. Każdy stan odpowiada prefiksowi słów z \( S^+ \), a przejścia są etykietowane symbolami \( H \) lub \( T \). Wynik tego kroku można zobaczyć na rysunku \ref{fig:alergia_example_0}. 

\begin{figure}[ht]
    \centering
    \includegraphics[width=0.8\textwidth]{images/run_example/alergia/0.png}
    \caption{Drzewo prefiksowe (PTA) skonstruowane dla \( S^+ \).}
    \label{fig:alergia_example_0}
\end{figure}

\paragraph*{Krok 2: Scalanie stanów \( q_1 \) i \( q_2 \).}  
Algorytm rozpoczyna proces scalania stanów od analizy zgodności pierwszej pary stanów \( q_1 \) i \( q_2 \). Weryfikacja odbywa się na podstawie rozkładów prawdopodobieństw przejść oraz liczby zakończeń w obu stanach.  

\paragraph*{Dane stanów:}  
\begin{itemize}  
    \item \( q_1 \): \( n = 8, F = 0 \), przejścia: \( H[5], T[3] \).  
    \item \( q_2 \): \( n = 5, F = 2 \), przejścia: \( H[2], T[1] \).  
\end{itemize}  

\paragraph*{Sprawdzenie terminacji:}  
Prawdopodobieństwo terminacji dla każdego stanu:  
\[
P_F(q_1) = \frac{0}{8} = 0, \quad P_F(q_2) = \frac{2}{5} = 0.4.
\]

Różnica:  
\[
|P_F(q_1) - P_F(q_2)| = |0 - 0.4| = 0.4.
\]

\paragraph*{Tolerancja Hoeffding’a:}  
Przy poziomie ufności \( \alpha = 1.25 \) oraz liczbach obserwacji \( n_1 = 8 \) i \( n_2 = 5 \), granica zgodności \( \epsilon \) jest obliczana na podstawie wzoru:  
\[
\epsilon = \sqrt{\frac{\ln(2 / \alpha)}{2}} \left( \frac{1}{\sqrt{n_1}} + \frac{1}{\sqrt{n_2}} \right).
\]

\paragraph*{Podstawienie wartości:}  
\[
\epsilon = \sqrt{\frac{\ln(2 / 1.25)}{2}} \left( \frac{1}{\sqrt{8}} + \frac{1}{\sqrt{5}} \right).
\]

Najpierw obliczamy logarytm:  
\[
\ln\left(\frac{2}{1.25}\right) = \ln(1.6) \approx 0.470.
\]

Podstawiamy:  
\[
\epsilon = \sqrt{\frac{0.470}{2}} \left( \frac{1}{\sqrt{8}} + \frac{1}{\sqrt{5}} \right).
\]

\[
\epsilon = \sqrt{0.235} \left( \frac{1}{2.828} + \frac{1}{2.236} \right).
\]

\[
\epsilon \approx 0.485 \left( 0.354 + 0.447 \right).
\]

\[
\epsilon \approx 0.485 \cdot 0.801 \approx 0.389.
\]

\paragraph*{Porównanie:}  
Ponieważ:  
\[
0.4 > 0.389,
\]  
stany \( q_1 \) i \( q_2 \) **nie mogą zostać połączone** bez dalszej analizy ich następców.  

\paragraph*{Rekurencyjna zgodność:}  
Aby kontynuować proces, algorytm przeprowadza sprawdzanie zgodności dla następujących par stanów:  
\begin{itemize}  
    \item \( q_1 \to q_3 \) i \( q_2 \to q_5 \).  
    \item \( q_2 \to q_4 \) i \( q_4 \to q_8 \).  
\end{itemize}  

Po dodatkowej analizie Hoeffding’a, wszystkie pary okazują się zgodne, co umożliwia scalanie całych podstruktur.  

\paragraph*{Rekurencyjne sprawdzanie następników:}  
Aby kontynuować analizę, algorytm porównuje następujące pary stanów:  
\begin{itemize}  
    \item \( q_1 \to q_3 \) i \( q_2 \to q_5 \).  
    \item \( q_2 \to q_4 \) i \( q_4 \to q_8 \).  
\end{itemize}  

Na podstawie testu Hoeffding’a wszystkie te pary okazują się zgodne, co pozwala na scalanie całych podstruktur.  

\paragraph*{Wynik:}  
Po scaleniu stanów \( q_1, q_2, q_4, q_8 \) oraz \( q_3, q_5 \) powstają dwa nowe stany:  
\begin{itemize}  
    \item \( q_1 \): \( n = 16, F = 4 \), przejścia: \( H[8], T[4] \).  
    \item \( q_3 \): \( n = 4, F = 2 \), przejścia: \( H[1], T[1] \).  
\end{itemize}  

Wynik tego kroku przedstawiono na rysunku \ref{fig:alergia_example_1}.  

\begin{figure}[ht]
    \centering
    \includegraphics[width=0.8\textwidth]{images/run_example/alergia/1.png}
    \caption{Graf po scaleniu stanów \( q_1, q_2, q_4, q_8 \) oraz \( q_3, q_5 \).}
    \label{fig:alergia_example_1}
\end{figure}  

