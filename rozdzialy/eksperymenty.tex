\chapter{Eksperymenty}
\label{cha:eksperymenty}

% opisz jakie implementacje używasz

% Mimo swoich zalet, \textit{ALERGIA} zazwyczaj nie jest odpowiednim wyborem do zastąpienia algorytmów \textit{GIG}, \textit{RPNI} lub \textit{L*}. Algorytm zakłada wyłącznie przykłady pozytywne, co ogranicza jego zdolność do wykrywania granic języka i może prowadzić do zbyt ogólnych modeli. Algorytmy takie jak \textit{RPNI} uwzględniają także przykłady negatywne, co pozwala na precyzyjniejsze modelowanie. Ponadto, \textit{ALERGIA} jest zoptymalizowana pod kątem modeli probabilistycznych, co czyni go mniej odpowiednim w przypadkach, gdzie wymagane są deterministyczne gramatyki lub klasyczne modele bez prawdopodobieństw, jak \textit{GIG} i \textit{L*}. Dlatego \textit{ALERGIA} sprawdza się głównie w zadaniach probabilistycznych, gdzie brak przykładów negatywnych, natomiast w kontekście klasycznych gramatyk regularnych bardziej odpowiednie pozostają algorytmy \textit{GIG}, \textit{RPNI} i \textit{L*}.