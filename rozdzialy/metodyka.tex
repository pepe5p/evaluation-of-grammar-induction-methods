\chapter{Metodyka}  
\label{cha:metodyka}  

Celem pracy jest analiza zachowania czterech algorytmów indukcji gramatyk formalnych: \textit{RPNI}, \textit{L*}, \textit{ALERGIA} oraz \textit{GIG}. Badania koncentrują się na ocenie ich wydajności, dokładności oraz zdolności do uogólniania danych. W tym celu przygotowane zostanie środowisko testowe umożliwiające przeprowadzanie eksperymentów oraz modyfikację parametrów testowych. Implementacja algorytmów zostanie zrealizowana w języku Python z wykorzystaniem bibliotek \textit{aalpy}, \textit{pygad} \cite{Gad2024}, \textit{numpy}, \textit{automata-lib} oraz \textit{pydot}.  

Python został wybrany jako środowisko implementacji ze względu na dostępność narzędzi wspierających pracę z automatami skończonymi, algorytmami genetycznymi oraz obliczeniami numerycznymi. Jego dynamiczna składnia oraz bogata dokumentacja ułatwiają szybkie prototypowanie algorytmów i modyfikację kodu, co jest kluczowe podczas projektowania eksperymentów.  

Algorytmy \textit{RPNI}, \textit{L*} i \textit{ALERGIA} zostaną zaimplementowane z wykorzystaniem biblioteki \textit{aalpy}, która oferuje gotowe funkcje do pracy z automatami skończonymi oraz umożliwia łatwą konfigurację parametrów testowych. Algorytm \textit{GIG} zostanie opracowany od podstaw przy użyciu bibliotek \textit{numpy} i \textit{pygad}, które wspierają algorytmy genetyczne, w tym definiowanie funkcji oceny (\textit{fitness functions}), krzyżowanie i mutacje. Do analizy i wizualizacji automatów wykorzystane zostaną biblioteki \textit{automata-lib} oraz \textit{pydot}, zapewniające czytelne przedstawienie wynikowych modeli.  

Dane testowe będą syntetyczne, dostosowane do specyfiki każdego algorytmu. Część z nich zostanie wygenerowana ręcznie, a część automatycznie, co umożliwi kontrolę nad parametrami, takimi jak rozmiar alfabetu, liczba przykładów oraz poziom zakłóceń. Takie podejście zapewni powtarzalność wyników oraz analizę wpływu zmiennych na jakość modeli.

Podsumowując, praca koncentruje się na ocenie skuteczności i wydajności algorytmów indukcji gramatyk formalnych w różnych warunkach. Wyniki pozwolą na porównanie ich możliwości oraz wskazanie potencjalnych zastosowań. W przyszłości badania mogą zostać rozszerzone o dane rzeczywiste oraz dodatkowe algorytmy, co umożliwi bardziej kompleksową analizę metod indukcji gramatyk.
