\chapter{Metodyka}
\label{cha:metodyka}

% Opis metod, technik lub narzędzi użytych do realizacji projektu.
% Szczegółowe uzasadnienie wyboru określonych metod.
% Czasem umieszcza się również schematy blokowe, diagramy lub modele stosowane w badaniu.

W ramach pracy zbudowano środowisko do ewaluacji metod indukcji gramatyk, umożliwiające porównanie różnych technik pod kątem ich wydajności i zastosowań praktycznych. Zaprojektowane eksperymenty koncentrują się na analizie efektywności wybranych algorytmów w różnych scenariuszach testowych oraz na zbadaniu ich odporności na specyficzne wyzwania związane z danymi wejściowymi. W pracy uwzględniono cztery algorytmy: \textit{RPNI}, \textit{L*}, \textit{ALERGIA} oraz \textit{GIG}. Eksperymenty zostały przeprowadzone dla każdego z nich według określonych kryteriów oceny wyników.

Dla algorytmu \textit{RPNI} analizowano złożoność czasową oraz dokładność wynikowego modelu w zależności od rozmiaru alfabetu. W przypadku algorytmu \textit{L*} mierzono liczbę zapytań EQ i MQ oraz rozmiar wygenerowanego automatu w zależności od złożoności problemu. Algorytm \textit{ALERGIA} oceniano pod względem odporności na szum w danych, mierząc procent poprawnie sklasyfikowanych zdań oraz liczbę stanów w wynikowym automacie. Dla algorytmu \textit{GIG} analizowano dokładność i wydajność w zależności od parametrów populacji początkowej, takich jak liczba generacji, rozmiar populacji oraz czas działania algorytmu. 

Dane użyte w eksperymentach były syntetyczne i zostały dobrane pod kątem specyfiki poszczególnych testów. Pozwoliło to na precyzyjne sterowanie warunkami eksperymentalnymi, takimi jak rozmiar alfabetu, poziom szumu czy liczność danych. Ocena wyników opierała się na różnych metrykach dostosowanych do analizowanych algorytmów. W przypadku \textit{RPNI} były to czas działania oraz liczba stanów wynikowego automatu. Dla \textit{L*} mierzono liczbę zapytań EQ i MQ oraz rozmiar automatu. W przypadku \textit{ALERGIA} analizowano procent poprawnych klasyfikacji z wygenerowanych przykładów i liczbę stanów w modelu wynikowym, natomiast dla \textit{GIG} oceniano czas działania, liczbę generacji, rozmiar automatu oraz dokładność klasyfikacji.

Eksperymenty przeprowadzono w środowisku Python 3.13 z wykorzystaniem odpowiednich bibliotek programistycznych. Algorytmy \textit{ALERGIA}, \textit{L*} i \textit{RPNI} zaimplementowano z użyciem biblioteki \textit{aalpy}, natomiast dla \textit{GIG} zastosowano \textit{automata-lib} (do konstrukcji automatów) oraz \textit{NumPy} i \textit{PyGAD} \cite{Gad2024} do operacji związanych z algorytmami genetycznymi. Do wizualizacji automatów wykorzystano bibliotekę \textit{pydot}.

Eksperymenty zostały zaprojektowane tak, aby umożliwić ich łatwą replikację. Szczegółowe konfiguracje parametrów dla każdego algorytmu opisano w rozdziale \ref{cha:eksperymenty}. Kod źródłowy oraz dane testowe zostaną udostępnione jako załączniki do pracy, co umożliwi ich ponowne wykorzystanie i weryfikację wyników. Badania koncentrowały się na przypadkach syntetycznych, co pozwoliło na precyzyjne modelowanie warunków testowych, jednak może ograniczać możliwość uogólnienia wyników na dane rzeczywiste.

Podsumowując, metodyka pracy obejmuje systematyczne podejście do testowania algorytmów indukcji gramatyk formalnych. Uwzględniono różne scenariusze eksperymentalne i wskaźniki oceny wyników, aby możliwe było pełne porównanie skuteczności badanych metod. W kolejnych rozdziałach szczegółowo opisano strukturę testów oraz analizę uzyskanych rezultatów.
