\chapter{Wprowadzenie}
\label{cha:wprowadzenie}

Odkrywanie gramatyki języka formalnego stanowi jedną z gałęzi uczenia maszynowego, mającą na celu znalezienie ustrukturyzowanego modelu, tj. gramatyki lub automatu, dla zaprezentowanego korpusu danych. Celem projektu jest zbudowanie środowiska do ewaluacji technik należących do tej dziedziny. W ramach prac przeprowadzona zostanie ewaluacja i implementacja wybranych algorytmów.

%---------------------------------------------------------------------------

\section{Cele pracy}
\label{sec:celePracy}


Celem poniższej pracy jest zapoznanie studentów z systemem \LaTeX~w zakresie umożliwiającym im samodzielne, profesjonalne złożenie pracy dyplomowej w systemie \LaTeX.

\subsection{Jakiś tytuł}

\subsubsection{Jakiś tytuł w subsubsection}


\subsection{Jakiś tytuł 2}

%---------------------------------------------------------------------------

\section{Zawartość pracy}
\label{sec:zawartoscPracy}

W rozdziale~\ref{cha:pierwszyDokument} przedstawiono podstawowe informacje dotyczące struktury dokumentów w \LaTeX u. Alvis~\cite{Alvis2011} jest językiem\textellipsis

Jeszcze kilka odnośników do bibliografii \cite{PeDa04,BuDo03}.