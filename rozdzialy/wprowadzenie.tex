\chapter{Wprowadzenie}  
\label{cha:wprowadzenie}  

Każdy internauta musiał kiedyś nauczyć się, jak wygląda poprawny adres e-mail. Początkowo użytkownik wpisywał różne frazy w formularzu rejestracyjnym, dowiadując się, czy dana sekwencja znaków jest akceptowana przez system. Przykładowo, mógł sprawdzić, że \texttt{jan.kowalski@example.com} zostaje zaakceptowany, podczas gdy \texttt{jan.kowalski@com} lub \texttt{jan.kowalski.example.com} są odrzucane. Mógł również znać kilka adresów, przykładowo swoich przyjaciół. Na podstawie odrzuconych formularzy oraz znanych już adresów e-mail użytkownik wykształcał pewną intuicję dotyczącą reguł rządzących poprawnymi adresami e-mail. Ten proces nauki jest przykładem indukcji gramatyki regularnej. 

Uczenie indukcyjne (\textit{inductive learning}) jest jednym z podstawowych podejść w dziedzinie uczenia maszynowego, którego celem jest odnalezienie ogólnych reguł na podstawie zbioru przykładów. Proces ten polega na analizie dostępnych danych wejściowych w celu wykrycia ukrytych wzorców i struktur, które mogą zostać uogólnione do postaci formalnego modelu. Modele takie znajdują zastosowanie w szerokim zakresie problemów, od klasyfikacji i przewidywania po modelowanie języków formalnych i systemów dynamicznych.  

Jednym z obszarów zastosowania uczenia indukcyjnego jest indukcja gramatyk formalnych, która stanowi istotne zagadnienie zarówno w uczeniu maszynowym, jak i teorii automatów. Jej celem jest odtworzenie reguł opisujących język formalny na podstawie skończonego zbioru danych wejściowych. Proces ten pozwala na budowę modeli, takich jak automaty skończone lub gramatyki formalne, które mogą zostać zastosowane do analizy danych sekwencyjnych, rozpoznawania wzorców, przetwarzania języka naturalnego oraz weryfikacji systemów.  

Głęboką podstawą teoretyczną dziedziny indukcji gramatyk jest koncepcja identyfikacji w granicy (\textit{identification in the limit}) \cite{GOLD1967447}. Określa ona warunki, w których algorytm jest zdolny do poprawnej rekonstrukcji struktury języka na podstawie skończonej liczby przykładów. Chociaż wyniki teoretyczne dostarczają istotnych gwarancji dotyczących poprawności algorytmów, praktyczne aspekty indukcji gramatyk pozostają przedmiotem dalszych badań, zwłaszcza w kontekście danych rzeczywistych, które często są obarczone szumem lub są niepełne.

%---------------------------------------------------------------------------

\section{Cele pracy}
\label{sec:celePracy}

% Szczegółowe określenie celów badania lub projektowania.
% Zakres pracy, czyli określenie granic projektu (co zostanie wykonane, a co nie).
% Problemy, które praca ma rozwiązać.  

Celem niniejszej pracy jest ewaluacja i porównanie wybranych algorytmów indukcji gramatyk formalnych. Przeanalizowane zostaną zarówno algorytmy klasyczne, takie jak \textit{L*} i \textit{RPNI}, jak i bardziej zaawansowane podejścia, w tym probabilistyczny algorytm \textit{ALERGIA} oraz metoda oparta na algorytmach genetycznych (\textit{GIG}). Badania koncentrują się na dokładności generowanych modeli, ich efektywności obliczeniowej oraz zdolności do uogólniania informacji przy niepełnych lub zaszumionych danych.

Przedstawione badania mają na celu nie tylko porównanie istniejących algorytmów, ale również dostarczenie praktycznych wskazówek dotyczących ich zastosowania w zależności od charakterystyki danych i wymagań aplikacyjnych. Praca stanowi krok w kierunku lepszego zrozumienia możliwości oraz ograniczeń metod indukcji gramatyk, otwierając jednocześnie perspektywy dla dalszych badań w tym obszarze.  

W ramach pracy zbudowane zostanie środowisko do ewaluacji metod indukcji gramatyk, umożliwiające porównanie różnych technik pod kątem ich wydajności i zastosowań praktycznych. Eksperymenty obejmują analizę danych syntetycznych, a wyniki zostaną ocenione pod względem jakości generowanych modeli i złożoności czasowej. 

%---------------------------------------------------------------------------

\section{Zawartość pracy}
\label{sec:zawartoscPracy}

% Przegląd literatury (teoretyczne podstawy pracy)
% Omówienie istniejących prac i badań związanych z tematem.
% Podstawy teoretyczne, które są niezbędne do zrozumienia tematu.
% Analiza istniejących rozwiązań, narzędzi lub metod stosowanych w dziedzinie.
% Krótki przegląd struktury pracy.

Niniejsza praca jest poświęcona analizie oraz porównaniu algorytmów indukcji gramatyk regularnych. Jej celem jest przedstawienie teoretycznych podstaw, metodologii oraz wyników eksperymentalnych dotyczących wybranych algorytmów. Praca składa się z sześciu rozdziałów, które szczegółowo opisano poniżej.

Pierwszy rozdział stanowi wprowadzenie do tematyki pracy. Przedstawiono w nim ogólne informacje dotyczące problematyki indukcji gramatyk formalnych. Omówiono cele pracy oraz zarysowano jej strukturę, wskazując na kluczowe aspekty analizowanych algorytmów. 

Drugi rozdział opisuje metodykę badań, w tym kryteria wyboru algorytmów oraz przyjęte założenia eksperymentalne. Zawiera również szczegółowy opis sposobu oceny skuteczności analizowanych metod oraz charakterystykę danych użytych w eksperymentach. 

Trzeci rozdział jest najobszerniejszą częścią pracy i zawiera szczegółowe opisy wybranych algorytmów. Przedstawiono w nim cztery kluczowe metody: \textit{L*}, \textit{RPNI}, \textit{GIG} oraz \textit{ALERGIA}. Każdy z tych algorytmów został opisany pod kątem swojej metody działania, formalizacji oraz zilustrowany praktycznymi przykładami zastosowań. Omówiono również różnice pomiędzy podejściem genetycznym a zachłannym, które są kluczowe w kontekście algorytmu \textit{GIG}. Dodatkowo przedstawiono krótki przegląd pozostałych algorytmów, które nie zostały uwzględnione w szczegółowej analizie.

Czwarty rozdział skupia się na eksperymentach przeprowadzonych w celu oceny skuteczności opisanych algorytmów. Opisano specyfikę eksperymentów, zestawy danych oraz metryki służące do oceny wyników.

W piątym rozdziale omówiono wyniki eksperymentów. Przeprowadzono analizę uzyskanych rezultatów, porównując algorytmy pod kątem ich zdolności do generalizacji, jakości wyników oraz efektywności obliczeniowej. Przedstawiono również ograniczenia poszczególnych metod oraz wskazano potencjalne kierunki ich ulepszenia. Zawiera podsumowanie pracy oraz wnioski płynące z przeprowadzonych badań.

Przegląd literatury, podstawy teoretyczne oraz analiza istniejących rozwiązań zostały zintegrowane z poszczególnymi rozdziałami, co zapewnia spójność tematyczną i logiczną narrację całej pracy.
