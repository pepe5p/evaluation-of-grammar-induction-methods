\chapter{Wprowadzenie}
\label{cha:wprowadzenie}

Odkrywanie gramatyki języka formalnego stanowi jedną z gałęzi uczenia maszynowego, mającą na celu znalezienie ustrukturyzowanego modelu, tj. gramatyki lub automatu, dla zaprezentowanego korpusu danych. Celem projektu jest zbudowanie środowiska do ewaluacji technik należących do tej dziedziny. W ramach prac przeprowadzona zostanie ewaluacja i implementacja wybranych algorytmów.
% wspomnieć o identification in the limit i jego konsekwencjach + coś z prezki Duponta

%---------------------------------------------------------------------------

\section{Cele pracy}
\label{sec:celePracy}

% Szczegółowe określenie celów badania lub projektowania.
% Zakres pracy, czyli określenie granic projektu (co zostanie wykonane, a co nie).
% Problemy, które praca ma rozwiązać.

%---------------------------------------------------------------------------

\section{Zawartość pracy}
\label{sec:zawartoscPracy}

% Przegląd literatury (teoretyczne podstawy pracy)
% Omówienie istniejących prac i badań związanych z tematem.
% Podstawy teoretyczne, które są niezbędne do zrozumienia tematu.
% Analiza istniejących rozwiązań, narzędzi lub metod stosowanych w dziedzinie.

% Przedstawienie kontekstu pracy.
% Uzasadnienie wyboru tematu.
% Cel pracy oraz opis problemu badawczego.
% Krótki przegląd struktury pracy.
